\documentclass[a4paper, 10pt]{article}
\usepackage[utf8]{inputenc}
\usepackage{xspace} % Needed for \xspace
\usepackage{hyperref} % Needed for \href
\usepackage{graphicx} % Needed for \includegraphics
\usepackage{float} % Needed for figures with the H placement specifier

% Utility macros
\newcommand{\atweb}{\textbf{@Web}\xspace}
\newcommand{\code}[1]{\texttt{#1}}
\newcommand{\img}[3]{
  \begin{figure}[H]
    \begin{center}
      \includegraphics[width=#3]{img/#1}
      \caption{#2}
    \end{center}
  \end{figure}
}

\begin{document}

\title{
  \atweb Constraint Checking: \\
  Functional Specification
}
\author{
  Leandro Lovisolo
  \texttt{\href{mailto:leandro.lovisolo@supagro.inra.fr}
               {<leandro.lovisolo@supagro.inra.fr>}}, \\
  INRA SupAgro and INRIA GraphiK, \\
  Montpellier, France
}
\date{December 2015}

\maketitle

\begin{abstract}
  In this document I summarize the changes proposed to the \atweb platform in
  order to implement automatic constraint checking using SPARQL queries.
\end{abstract}

%%%%%%%%%%%%%%%%%%%%%%%%%%%%%%%%%%%%%%%%%%%%%%%%%%%%%%%%%%%%%%%%%%%%%%%%%%%%%%%%

\section{Core ontology changes}

The following list provides a high level view of the changes proposed to the
core ontology.

\begin{itemize}
  \item A new OWL class \code{Constraint} is added to the core ontology,
  which represents constraints expressed as SPARQL queries.

  \item Instances of the \code{Constraint} class are associated to their
  respective relation classes via a new object property
  \code{hasForConstraint}.

  \item A new data property \code{hasForSPARQLQuery} is added, which connects
  instances of the \code{Constraint} class with a string literal holding the
  actual SPARQL query.

  \item Constraints are described in natural language with a textual guideline
  associated to \code{Constraint} instances via a SKOS scope note.

  \item A new \code{Error} concept is added, with the purpose of storing
    validation errors collected during the execution of the SPARQL constraints.

  \item Instances of the \code{Error} class are linked to the \code{Constraint}
    instance that produced the validation error via a new object property
    \code{hasForConstraint}.

  \item Instances of the \code{Error} class are linked to the \code{Row}
    instance associated to the unit operation relation instance that caused the
    error via a new object property \code{hasForRow}.
\end{itemize}

\subsection{Biorefinery domain ontology}

This particular domain requires the notion of experiment \textit{categories},
which have associated constraints that require additional information. In the
following sections, the concept of categories is explained, and then some
changes to the domain ontology are proposed to support expressing the required
constraints as SPARQL queries.

\subsubsection{Categories}

Categories are a way to group unit operation relation instances according to
the kind of experiment they model. This grouping is currently done in the
\atweb platform via document topics (e.g., Bioref-PM, Bioref-PM-UFM,
Bioref-PM-PC-EX-PS, etc.)

Categories are characterized by the two following points:

\begin{itemize}
  \item Each unit operation relation instance must belong to exactly one
    category.

  \item Each category has clearly defined rules to decide whether an experiment
    (i.e. a set of unit operation relation instances) belongs to it or not.
\end{itemize}

Thus, given a unit operation relation instance and the category it belongs to,
it is required to check said rules automatically. To this end, rules are
encoded as constraints written as SPARQL queries.

\subsubsection{Changes to the Biorefinery domain ontology}

A new symblic concept \code{ProcessType} is created, with one subclass for each
supported category. Each such subclass is listed below, with its proposed
alternate label between parentheses:

\begin{itemize}
  \item \code{Milling} (PM)
  \item \code{Milling\_PhysicoChemical\_Extrusion} (PM-PC-EX-PS)
  \item \code{Milling\_PhysicoChemical} (PM-PC-PS)
  \item \code{Milling\_PhysicoChemical\_UltraFineMilling} (PM-PC-UFM)
  \item \code{Milling\_PhysicoChemical\_UltraFineMilling\_PressSeparation}
  (PM-PC-UFM-PS)
  \item \code{Milling\_UltraFineMilling} (PM-UFM)
\end{itemize}

A new argument is added to the n-ary relations that represent unit operations
in this domain with the purpose of linking a relation instance with the
category (i.e. \code{ProcessType} subclass) it belongs to.

The \code{Constraint} instances that verify the category inclusion rules are
associated to the \code{Relation} class in order to make them available to all
subclasses (i.e. all relations).

In a future version of \atweb there should be a mechanism for ontology-level
constraints that would allow a more correct way of expressing category
constraints.

%%%%%%%%%%%%%%%%%%%%%%%%%%%%%%%%%%%%%%%%%%%%%%%%%%%%%%%%%%%%%%%%%%%%%%%%%%%%%%%%

\section{Uploading constraints in CSV format}

A new screen is added to the \atweb management UI with the purpose of uploading
constraints as CSV files. Such files would have the following columns:

\begin{itemize}
  \item \code{prefLabel=EN} (e.g. \textit{Milling})
  \item \code{altLabel=EN} (e.g. \textit{PM})
  \item \code{scopeNote=EN} (textual guideline)
  \item \code{Relation\_Concept}
  \item \code{SPARQL\_query}
\end{itemize}

%%%%%%%%%%%%%%%%%%%%%%%%%%%%%%%%%%%%%%%%%%%%%%%%%%%%%%%%%%%%%%%%%%%%%%%%%%%%%%%%

\section{User interface}

The constraint visualization and verification user interface is described in
the following sections.

\subsection{Constraint visualization}

Constraints are displayed next to the scope notes when exploring relations, as
shown below.

\img{constraint-visualization.png}{Constraint visualization.}{12cm}

Note that this visualization is read-only.

\subsection{Constraint verification}

The constraint verification process is done on a per-table basis, and is
launched by right-clicking the target table and selecting \textit{validate
constraints}, as shown below.

\img{validate-constraints-menu-item.png}{
  \textit{validate constraints} menu item.
}{5cm}

The user is then taken to a screen where they can select the constraints to
verify. This list is compiled by finding all constraints linked to the
categories associated with all relation instances in the table annotations.

\img{constraint-selection.png}{Constraint selection.}{12cm}

After clicking the \textit{Validate selected constraints} button, a loading
indicator is shown. When the queries finish running the user is taken to a
results screen sketched below, where a summary of the errors is provided.

\img{validation-results-summary.png}{Summary of validation results.}{8cm}

The table icon is updated to reflect the presence or absence of errors.

\img{table-pin-error.png}{
  Table icon changes to reflect the presence of errors.
}{5cm}

In the case of errors, when the user enters the table edition screen they will
see the rows in a table containing errors highlighted in red.

\img{edit-annotation.png}{Annotated table with validation errors.}{12cm}

The user can get additional information (such as the name of the constraint
violated) by hovering the mouse pointer over the affected row.

\img{validation-results-details.png}{Validation error details.}{12cm}

After performing modifications on an affected row, the row color goes back to
gray.

\img{validation-results-after-modifications.png}{
  Row with validation errors no longer in red after modifying its contents.
}{12cm}

The table icon also changes color to reflect this new state, i.e. the table
used to have errors and it has since been modified, but its validation
constraints have yet to be rerun.

\img{table-pin-after-modifications.png}{
  Table icon after modifying its annotations in the presence of validation
  errors.
}{5cm}

If no errors are found after running the validation constraints, the table icon
should be reset to the default icon.

%%%%%%%%%%%%%%%%%%%%%%%%%%%%%%%%%%%%%%%%%%%%%%%%%%%%%%%%%%%%%%%%%%%%%%%%%%%%%%%%

\section{Other implementation details}

\code{Error} instances should be stored in the annotations ontology.

An \code{Error} instance should be eliminated after a modification to its
related row is performed.

A new table should be created in the relational database to keep a list of all
annotated tables that should be shown in the UI with an icon indicating the
presence of errors. That is, annotated tables that have or had validation
errors and have been modified since those errors were found, but their
validation constraints haven't been rerun.

\end{document}
